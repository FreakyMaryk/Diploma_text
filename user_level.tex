\section {Пользовательский уровень сборщика мусора JBGC}

\subsection {Использование JBGC}
Для того, чтобы воспользоваться сборщиком мусора JBGC нужно сначала установить его на ваш компьютер. Сборщик мусора изначально написан под Ubuntu. Работает он в системах семейства Linux. Для того, чтобы его установить, нужно выполнить последовательно следующие команды в терминале:
\begin{enumerate}
\item ./autogen.sh
\item ./configure
\item make clean | make | make install
\end {enumerate}

При запуске программы обязательно указывать переменную окружения LD\_LIBRARY\_PATH \footnote{http://www.opennet.ru/base/dev}, список отделённых  двоеточиями  имён каталогов, где библиотеку JBGC -- gc следует искать перед  тем,  как  их  будут  искать  по стандартным путям.

Программа, использующая сборщик мусора JBGC должна удовлетворять следующим ограничениям:
\begin{enumerate}
\item Все указатели  должны быть заменены на  объекты класса gc\_ptr -- gc\_ptr<typename> \begin{lstlisting}int* a; ->  gc_ptr<int> a;
\end{lstlisting}
\item Все вызовы new заменены на gc\_new<typenames>(params)(size\_t size) \begin{lstlisting} new int [5]; -> gc_new<int>(5);
\end{lstlisting}
\end {enumerate}
А также нужно сделать include заголовочного файла:\begin{lstlisting}
#include <libgc/libgc.h>;
\end{lstlisting}

Небольшая программка, использующая сборщик мусора JBGC:
\begin{lstlisting}
#include <cstdio>
#include <libgc/libgc.h>

class point {
	int x,y
}

int main() {
	gc_ptr<point> pt;
	gc_ptr<int> p;

	pt = gc_new<point>();
	p = gc_new<int>();
}

\end{lstlisting}
\subsection {Возможности ``умного указателя'' gc\_ptr}
В следствие одного из главных требований, накладываемых на реализацию сборщика мусора:  не должно быть никаких операций над указателями (логических, арифметических и т.п.), был реализован набор операторов для работы с gc\_ptr:

\begin{enumerate}
\item (*) или разыменовывания -- примененная к gc\_ptr, обеспечивает доступ к содержимому ячейки памяти, на которую ссылается указатель. 
Пример использования:
\begin{lstlisting}
int a = 5, b = 6;
gc_ptr<int> ar = &a, br = &b;
* ar = 10;
* br = *ar;
\end{lstlisting}

\item  (->) или стрелка, обеспечивает доступ к элементам структур.

Пример:
\begin{lstlisting}
struct A {
	int a;
	int b;
}
/* создается gc_ptr на структуру A */
gc_ptr<A> obj = gc_new<A>();
/* обращение к полям структуры */
obj -> a = 5;
obj -> b = 6;
\end{lstlisting}

\item ([ ]) или обращения к элементу. 

Пример:
\begin{lstlisting}
/* создается массив элементов типа int длины len */
gc_ptr<int> ar = gc_new<int>(len), br = gc_new<int>(len);
	for (int i = 0; i < len; i++) {
		ar[i] = i;
		br[i] = len - i;
	}
\end{lstlisting}

\item (==) или проверки на равенство, вернет соответственно true и false, только если gc\_ptr указывают на один и тот же объект. 

Пример:
\begin{lstlisting}
int a = 5, b = 6;
int * ar2 = &a, br2 = &b;
gc_ptr<int> ar = &a, br = &b;
/* в этом случае результат будет false */
if (ar == br) {
...	
}
/* в этом случае результат будет true */
if (ar== ar2) {
...
}

/* в этом случае результат будет false */
if (ar == 0) {
...
}
\end{lstlisting}

\item (!=) или проверки на неравенство, вернет соответственно true и false, только если gc\_ptr указывают на разные объекты. 

Пример:
\begin{lstlisting}
int a = 5, b = 6;
int * ar2 = &a, br2 = &b;
с
/* результат будет true */
if (ar != br) {
...	
}
/* результат будет false */
if (ar != ar2) {
...
}
if (ar !=  0) {
...
}
\end{lstlisting}

\item (=) или присваивания. 

Пример:
\begin{lstlisting}
int a = 5;
int * ar2 = &a;
gc_ptr<int> ar = gc_new<int>();
gc_ptr<int> br = gc_new<int>();
ar = ar2;
br = ar;
\end{lstlisting}

\end {enumerate}

Представленных операций достаточно для того, чтобы реализовать основные примитивы языка С++.

\subsection{ Возможности функции выделения памяти gc\_new}

В языке С++ существует пять различных вариантов вызова функции выделения памяти для объекта --- new. Оператор пытается выделить достаточно памяти в куче для размещения новых данных и, в случае успеха, возвращает адрес свежевыделенной памяти. После выделения памяти вызывается конструктор объекта. Однако, если new не может выделить память в куче, то он передаст (throw) исключение типа std::bad\_alloc. Это устраняет необходимость явной проверки результата выделения. После встречи компилятором ключевого слова new им генерируется вызов конструктора класса.
\begin {enumerate}

\item \textit{new type} -- выделение памяти под обект типа type, кроме объектов структур, классов. 

Пример:
 \begin{lstlisting}
float* pf = new float;
\end{lstlisting}
 
\item \textit{new A()} -- выделение памяти под объект класса/структуры, после этого вызов конструктора по умолчанию. 

Пример:
 \begin{lstlisting}
class CName {
public:
   enum {
      sizeOfBuffer = 256
   };

};
int main() {
   CName* pName = new CName;
}
\end{lstlisting}

\item\textit{ new type [len]} -- выделение памяти под массив, состоящий из элементов типа type, длинной len.

 Пример:
 \begin{lstlisting}
  int* p_array = new int[5];
\end{lstlisting}

\item \textit{new A(p1, p2, ...)} -- выделение памяти под объект класса A, после этого вызов конструктора не по умолчанию. 

Пример:
 \begin{lstlisting}
class A {
private:
	int size;
	int * mas;
public:
	A () {}
	~A () {}

	A (int len, const int * & ar) {
		size = len;
		this -> mas =new  int [len];
		for (int i = 0; i < len; i++) {
			mas[i] = ar[i];
		}
	}
	};
int main () {
	A* a = new A (len, ar);
	A* b = new A (len, br);
	for (int i = 0; i < len; i++) {
		ar[i] += 10;
		br[i] += 100;
	}
	return 0;
}
\end{lstlisting}
\end {enumerate}

Для gc\_new существуют вызовы по-сути схожие стандартным вызовам new.
\begin {enumerate}
\item \textit{gc\_new<type>()} --  выделение памяти под обект типа type, кроме структур, классов. 
Пример:
 \begin{lstlisting}
gc_ptr<float> pf = gc_new <float>();
\end{lstlisting}
\item \textit{gc\_new< class\_name>()} -- выделение памяти под объект класса/структуры, после вызов конструктора по умолчанию. 
Пример:
 \begin{lstlisting}
class CName {
public:
   enum {
      sizeOfBuffer = 256
   };

};
int main() {
  gc_ptr<CName> pName = gc_new< CName>();
}
\end{lstlisting}

\item \textit{gc\_new <type>(len)} -- выделение памяти под массив из элементов типа type, c количеством элементов len. 
Пример:
 \begin{lstlisting}
 gc_ptr<int> p_array = gc_new <int>(5);
\end{lstlisting}

\item \textit{gc\_new <class\_name, types\_of\_objects...>(список аргументов конструктора не по умолчанию)} -- выделение памяти под обект класса/структуры. В треугольных скобках слева направо аргументы: class\_name -- имя класса/структуры объекта, под который отводится память; types\_of\_objects -- список типов объектов, присутствующих в конструкторе, строго в порядке указания в конструкторе; список имен объектов или значений, передаваемых конструктору(список аргументов в круглых скобках). 
Пример:
 \begin{lstlisting}
class A {
private:
	int size;
	gc_ptr<int> mas;
public:
	A () {}
	~A () {}

	A (int len, const gc_ptr<int> & ar) {
		size = len;
		this -> mas = gc_new<int>(len);
		for (int i = 0; i < len; i++) {
			mas[i] = ar[i];
		}
	}
};
int main () {
	gc_ptr<A> a = gc_new<A, int, gc_ptr<int>&>(len, ar);
	gc_ptr<A> b = gc_new<A, int, gc_ptr<int>&>(len, br);
	return 0;
}
\end{lstlisting}
\end {enumerate}

Таким образом процесс выделения памяти в языке С++ для указателей различных типов можно заменить вызовом  функции gc\_new соответствующего вида.

\subsection {Сравнительный пример без использования сборщика мусора и с использованием}
Теперь рассмотрим код написанные на `` чистом'' C++, без использования сборки мусора, и с использованием.
Для демонстрации примера использования JBGC было выбрано ``декартово дерево''. Декартово дерево (англ. cartesian tree, treap)\footnote{http://habrahabr.ru/post/101818/} — красивая и легко реализующаяся структура данных, которая с минимальными усилиями позволит производить многие скоростные операции над массивами данных. 
\begin{lstlisting}
#include <cstdio>
#include <algorithm>
#include <cassert>

using namespace std;

typedef unsigned int uint;

template <class T> 
class cartesian_tree
{ 
  public:
/* У декартова дерева есть следующие поля:
 * -- значение
 * -- приоритет
 * -- глубина дерева
 * -- левое поддерево
 * -- правое поддерево
*/
  T key;
  uint y;
  uint count;

  cartesian_tree *l, *r;

  cartesian_tree (int _key)
  {
    count = 1;
    key = _key;
    y = ((((uint)rand())<<15) + rand());
    l = NULL;
    r = NULL;
  }

  cartesian_tree (void)
  {
    count = 1;
    key = 0;
    y = 0;
    l = NULL;
    r = NULL;
  }
};

/* Операция Merge принимает на вход два декартовых дерева u и v. 
 * От нее требуется слить их в одно, тоже корректное, декартово дерево v.
*/
template <class T>
cartesian_tree<T> * merge (cartesian_tree<T> *u, cartesian_tree<T> *v)
{
  if (u == NULL)
    return v;

  if (v == NULL)
    return u;

  if (u->y >= v->y)
  {
    u->r = merge (u->r, v);
    u->count = (u->l != NULL ? u->l->count : 0) + (u->r != NULL ? u->r->count : 0) + 1;
    return u;
  }
  else
  {
    v->l = merge (u, v->l);
    v->count = (v->l != NULL ? v->l->count : 0) + (v->r != NULL ? v->r->count : 0) + 1;
    return v;
  }
}

/* Операция Split принимает на вход корректное декартово дерево uи некий ключ key. 
* Задача операции — разделить дерево на два так, 
 * чтобы в одном из них ( l ) оказались все элементы исходного дерева с ключами, меньшими key, а в другом ( r ) — с большими.
*/
template <class T>
cartesian_tree<T> split (cartesian_tree<T> *u, int key)
{
  cartesian_tree<T> res, tmp;

  res.l = res.r = NULL;

  if (u == NULL)
    return res;

  if (u->key >= key)
  {
    tmp = split (u->l, key);
    u->l = tmp.r;
    u->count = (u->l != NULL ? u->l->count : 0) + (u->r != NULL ? u->r->count : 0) + 1;
    res.l = tmp.l, res.r = u;
  }
  else
  {
    tmp = split (u->r, key);
    u->r = tmp.l;
    u->count = (u->l != NULL ? u->l->count : 0) + (u->r != NULL ? u->r->count : 0) + 1;
    res.r = tmp.r, res.l = u;
  }

  return res;
}

/* удаления значения key из дерева */
template <class T>
cartesian_tree<T>* del (cartesian_tree<T> *u, cartesian_tree<T> *p, cartesian_tree<T> *node, T key)
{
  if (u == NULL)
    return node;

  if (u->key == key)
  {
    cartesian_tree<T> *d;
    if (p == NULL)
      d = node, node = merge (node->l, node->r);
    else
    {
      if (p->r == u)
        d = p->r, p->r = merge(u->l, u->r);
      else
        d = p->l, p->l = merge(u->l, u->r);
    }
    return node;
  }

  if (u->key > key)
    return del(u->l, u, node, key);
  else
    return del(u->r, u, node, key);
  u->count = (u->l != NULL ? u->l->count : 0) + (u->r != NULL ? u->r->count : 0) + 1;
}

/* добавление одного дерева в другое, в зависимостир от приоритета y */
template <class T>
cartesian_tree<T>* add (cartesian_tree<T> *u, cartesian_tree<T> *v)
{
  if (u == NULL)
    return v;

  if (u->y > v->y)
  {
    if (u->key < v->key)
    {
      u->r = add(u->r, v);
      u->count = (u->l != NULL ? u->l->count : 0ll) + (u->r != NULL ? u->r->count : 0) + 1;
    }
    else
    {
      u->l = add(u->l, v);
      u->count = (u->l != NULL ? u->l->count : 0ll) + (u->r != NULL ? u->r->count : 0) + 1;
    }
    return u;
  }
  else
  {
    cartesian_tree<T> tmp;
    tmp = split (u, v->key);
    v->l = tmp.l, v->r = tmp.r;
    v->count = (v->l != NULL ? v->l->count : 0) + (v->r != NULL ? v->r->count : 0) + v->key;
    return v;
  }

  return NULL;
}

template <class T>
uint greater_count (cartesian_tree<T>* u, T key)
{
  if (u == NULL)
    return 0;

  uint res = 0;
  
  if (u->key >= key)
  {
    if (u->r != NULL)
      res += u->r->count;
    return res + 1 + greater_count(u->l, key);
  }
  else
    return greater_count(u->r, key);
}

/* поиск минимального значения в дереве */
template <class T>
T get_min (cartesian_tree<T>* v)
{
  if (v->l != NULL)
    return get_min(v->l);
  return v->key;
}

/* удаление дерева*/
template <class T>
void clear (cartesian_tree<T> *u)
{
  if (u == NULL)
    return;
  clear(u->l), clear(u->r);
}

/* класс myset на основе декартова дерева */
template <class T>
class myset
{
public:
  cartesian_tree<T> *node;

  myset()
  {
    node = NULL;
  }

  void insert (int k)
  {
    cartesian_tree<T> *v = new cartesian_tree<T>(k);
    node = add (node, v);
  }

  void erase (int k)
  {
    node = del (node, NULL, node, k);
  }

  uint greater_count (T k)
  {
    return ::greater_count (node, k);
  }

  T get_min ()
  {
    return ::get_min(node);
  }
};

/* тестируем myset, добавляем рандомные элементы в список */
void test (int STEP_COUNT, int ELEM_COUNT)
{
  for (int i = 0; i < STEP_COUNT; i++)
  {
    myset <int> s;
    for (int j = 0; j < ELEM_COUNT; j++)   
      s.insert(rand() + j);
  }
} 

int main (int argc, char *argv[])
{
	assert(argc == 3);
	int STEP_COUNT, ELEM_COUNT;

	sscanf (argv[1], "%d", &STEP_COUNT);
	sscanf (argv[2], "%d", &ELEM_COUNT);

	test(STEP_COUNT, ELEM_COUNT);

	return 0;
}
\end{lstlisting}

Далее приведен код с использование сборщика мусора. От предыдущего он отличается тем, что в нем добавлено использование стандартного map. И функции сравнения myset и map. Это необходимо для того, чтобы убедиться, что не было удалено лишних элементов.
\begin{lstlisting}
#include <cstdio>
#include <libgc/libgc.h>
#include <set>

using namespace std;

extern bool new_active;
 
typedef unsigned int uint;
 
template <class T>
class cartesian_tree
{
  public:
 
 
  T key;
  uint y;
  int count;
 
  gc_ptr <cartesian_tree<T> > l, r;
 
  cartesian_tree (T _key)
  {
    count = 1;
    key = _key;
    y = ((rand()<<15) + rand());
    l = 0;
    r = 0;
  }
 
  cartesian_tree (void)
  {
    count = 1;
    key = 0;
    y = 0;
    l = 0;
    r = 0;
  }
};
 
template <class T>
gc_ptr <cartesian_tree<T> > merge (gc_ptr <cartesian_tree<T> > u, gc_ptr <cartesian_tree<T> > v)
{
  if (u == 0)
    return v;
 
  if (v == 0)
    return u;
 
  if (u->y >= v->y)
  {
    u->r = merge (u->r, v);
    u->count = (u->l != 0 ? u->l->count : 0) + (u->r != 0 ? u->r->count : 0) + 1;
    return u;
  }
  else
  {
    v->l = merge (u, v->l);
    v->count = (v->l != 0 ? v->l->count : 0) + (v->r != 0 ? v->r->count : 0) + 1;
    return v;
  }
}
 
template <class T>
cartesian_tree<T> split (gc_ptr <cartesian_tree<T> > u, T key)
{
  cartesian_tree<T> res, tmp;
 
  res.l = res.r = 0;
 
  if (u == 0)
    return res;
 
  if (u->key >= key)
  {
    tmp = split (u->l, key);
    u->l = tmp.r;
    u->count = (u->l != 0 ? u->l->count : 0) + (u->r != 0 ? u->r->count : 0) + 1;
    res.l = tmp.l, res.r = u;
  }
  else
  {
    tmp = split (u->r, key);
    u->r = tmp.l;
    u->count = (u->l != 0 ? u->l->count : 0) + (u->r != 0 ? u->r->count : 0) + 1;
    res.r = tmp.r, res.l = u;
  }
 
  return res;
}
 
template <class T>
gc_ptr <cartesian_tree<T> > del (gc_ptr <cartesian_tree<T> >u, gc_ptr <cartesian_tree<T> > p, gc_ptr <cartesian_tree<T> > node, T key)
{
  if (u == 0)
    return node;
 
  if (u->key == key)
  {
    gc_ptr <cartesian_tree<T> > d;
    if (p == 0)
      d = node, node = merge (node->l, node->r);
    else
    {
      if (p->r == u)
        d = p->r, p->r = merge(u->l, u->r);
      else
        d = p->l, p->l = merge(u->l, u->r);
    }
    return node;
  }
 
  if (u->key > key)
    return del(u->l, u, node, key);
  else
    return del(u->r, u, node, key);
  u->count = (u->l != 0 ? u->l->count : 0) + (u->r != 0 ? u->r->count : 0) + 1;
}
 
template <class T>
gc_ptr <cartesian_tree<T> > add (gc_ptr <cartesian_tree<T> > u, gc_ptr <cartesian_tree<T> > v)
{
  if (u == 0)
    return v;
 
  if (u->y > v->y)
  {
    if (u->key < v->key)
    {
      u->r = add(u->r, v);
      u->count = (u->l != 0 ? u->l->count : 0) + (u->r != 0 ? u->r->count : 0) + 1;
    }
    else
    {
      u->l = add(u->l, v);
      u->count = (u->l != 0 ? u->l->count : 0) + (u->r != 0 ? u->r->count : 0) + 1;
    }
    return u;
  }
  else
  {
    cartesian_tree<T> tmp;
    tmp = split (u, v->key);
    v->l = tmp.l, v->r = tmp.r;
    v->count = (v->l != 0 ? v->l->count : 0) + (v->r != 0 ? v->r->count : 0) + v->key;
    return v;
  }
 
  return 0;
}
 
template <class T>
uint greater_count (gc_ptr <cartesian_tree<T> > u, T key)
{
  if (u == 0)
    return 0;
 
  uint res = 0;
 
  if (u->key >= key)
  {
    if (u->r != 0)
      res += u->r->count;
    return res + 1 + greater_count(u->l, key);
  }
  else
    return greater_count(u->r, key);
}
 
template <class T>
T get_min (gc_ptr <cartesian_tree<T> > v)
{
  if (v->l != 0)
    return get_min(v->l);
  return v->key;
}
 
template <class T>
void clear (gc_ptr <cartesian_tree<T> > u)
{
  if (u == 0)
    return;
  clear(u->l), clear(u->r);
}
 
template <class T>
class myset
{
public:
  gc_ptr <cartesian_tree<T> > node;
 
  myset()
  {
    node = 0;
  }
 
  void insert (T k)
  {
    gc_ptr <cartesian_tree<T> > v = gc_new <cartesian_tree<T>> (1);
    *v = cartesian_tree<T> (k);
    node = add (node, v);
  }
 
  void erase (T k)
  {
    node = del (node, gc_ptr<cartesian_tree<T>> (0), node, k);
  }
 
  uint greater_count (T k)
  {
    return ::greater_count (node, k);
  }
 
  T get_min ()
  {
    return ::get_min(node);
  }
};
 
/* Функция вставки и удаления элементов в multiset -- стандартную структуру С++
 * и в специальную myset с использование сборки мусора. 
 */
int main (void)
{
  int ELEM_COUNT = 100000;
  multiset<int> s;
  myset <int> ms;
  for (int i = 0; i < ELEM_COUNT; i++)
  {
    int k = ((rand()<<15) + rand());
    s.insert(k);
    ms.insert(k);
  }
  for (int j = 0;;j++)
  {
    for (int i = 0; i < 50000; i++)
    {

      if (ms.get_min() != *s.begin())
      {
        printf ("ERROR\n");
        return 0;
      }

      s.erase(s.begin());
      ms.erase(ms.get_min());
    }
    for (int i = 0; i < 50000; i++)
    {
      int k = ((rand()<<15) + rand());
      s.insert(k);
      ms.insert(k);
    }
  }
}
\end{lstlisting}