\section {Пользовательский уровень библиотеки\\
сборки мусора}

С точки зрения пользователя библиотека сборки мусора представлена одним шаблонным классом
``умного указателя'' 

\begin{lstlisting}
    template <class T> class gc_ptr {
       ...
    }
\end{lstlisting}

и одной шаблонной функцией выделения памяти

\begin{lstlisting}
    template <class T, typename ... Types> T* gc_new (
    Types ... types, size_t count = 1
    ) {
       ...
    }
\end{lstlisting}

которые становятся доступными после подключения одного заголовочного файла:

\begin{lstlisting}
    # include <libgc/libgc.h>
\end{lstlisting}

Исполняемый код библиотеки содержится в модулях, которые могут быть как собраны статически вместе
с приложением, так и подгружены динамически.

Любая программа, которая в качестве примитива выделения памяти использует \lstinline{gc_new}, в качестве
указателей --- \lstinline{gc_ptr}, и не содержит явного освобождения памяти с помощью \lstinline{delete}, будет
работать с использованием сборки мусора.

Семантика описанных примитивов с точки зрения сборки мусора будет описана в следующем разделе; пока же
мы обсудим их интерфейс с точки зрения пользователя.

\subsection{Интерфейс \lstinline{gc_ptr}}

Класс \lstinline{gc_ptr} инкапсулирует всю функциональность указателя, которая безопасна с точки зрения
сборки мусора. Данный класс реализует следующие операторы:

\begin{enumerate}
\item \lstinline{T& operator*() const}~--- разыменование;
\item \lstinline{T* operator->() const}~--- доступ к указателю;
\item \lstinline{T& operator[](size_t index)}~--- доступ к элементам массива;
\item \lstinline{bool operator== (const gc_ptr <T> &a), bool operator== (const T* a)}~--- проверки на равенство;
\item \lstinline{bool operator!= (const gc_ptr <T> &a), bool operator!= (const T* a)}~--- проверки на неравенство;
\item \lstinline{gc_ptr& operator = (const gc_ptr <T> &a), gc_ptr& operator = (T* a)}~--- присваивание.
\end{enumerate}

Кроме того, у данного класса два конструктора: \lstinline{gc_ptr (T* p)} и \lstinline{gc_ptr (const gc_ptr <T> &p)}. 

\subsection{Функция выделения памяти \lstinline{gc_new}}

В языке С++ существует пять различных вариантов выделения памяти с помощью оператора \lstinline{new}. 
Этот оператор пытается выделить достаточно памяти в куче для размещения новых данных и, в случае успеха, возвращает 
адрес выделенного участка. После выделения памяти вызывается конструктор создаваемого объекта. Однако, если оператор
\lstinline{new} не может выделить память в куче, то возбуждается исключение типа \lstinline{std::bad_alloc}. 

Функция \lstinline{gc_new} позволяет выразить все ситуации, в которых может быть употреблён оператор \lstinline{new}.
Ниже мы приведем примеры использования \lstinline{gc_new} в каждой из них.

\begin {enumerate}
\item \lstinline{gc_new<type> ()}~--- выделение памяти под значение атомарного типа (\lstinline{int}, \lstinline{float} и т.д.); 
\item \lstinline{gc_new<C> ()}~--- выделение памяти под одиночный экземпляр объект класса \lstinline{C} с вызовом конструктора по
умолчанию;
\item \lstinline{gc_new <type> (len)}~--- выделение памяти под массив длины \lstinline{len} из элементов атомарного типа \lstinline{type};
\item \lstinline{gc_new <C> (len)}~--- выделение памяти под массив длины \lstinline{len} из экземпляров класса \lstinline{C}, для каждого из
которых вызывается конструктор по умолчанию;
\item \lstinline{gc_new <C, T1, T2, ..> (a1, a2, ..)}~--- выделение памяти под экземпляр объекта класса \lstinline{C} с вызовом
конструктора с парамерами \lstinline{a1, a2, ..}, имеющими типы \lstinline{T1, T2, ..} соответственно.
\end {enumerate}

Таким образом примитив выделения памяти в языке С++ для объектов различных типов можно заменить вызовом функции \lstinline{gc_new} 
соответствующего вида.

\subsection{Пример использования пользовательских примитивов}

В качестве примера использования пользовательских примитивов библиотеки сборки мусора приведем 
реализацию класса строк:

\begin{lstlisting}
    class GCString {
      private:
        // указатель на массив символов
        gc_ptr<char> pData; 	
        // длина строки 
        int length;	
      public:
        GCString ();
        GCString (const char * cString);
        virtual ~GCString ();
        GCString (const GCString& s);
        GCString operator= (const GCString& s);
        GCString operator= ( const char* cString);
        char operator[] (int i);
        GCString operator+  (const GCString& s);
        GCString operator+  (const char * cString);
        GCString operator+= (GCString& s);
      private:
        void copy(const GCString& s);
    };

    GCString::GCString() {
      length = 0;
      pData  = NULL;
    }

    GCString::GCString(const char * cString) {
      length = strlen (cString);
      pData = gc_new <char> (length + 1);
      strcpy ((char*) pData, cString);
    }

    GCString::~GCString(){}

    GCString::GCString(const GCString& s) {
      copy (s);
    }

    GCString GCString::operator=(const GCString& s) {
      copy(s);
      return *this;
    }

    GCString GCString::operator=(const char * cString) {
      GCString temp (cString);
      return temp;
    }

    void GCString::copy(const GCString& s) {
      length = s.length;
      pData = gc_new <char> (length + 1);
      strcpy ((char*) pData, (char*) s.pData);
    }

    GCString GCString::operator+ (const GCString& s) {
      GCString cat;
      cat.length = length + s.length;
      cat.pData = gc_new <char> (cat.length + 1);
      for (int i = 0; i < length; i++) 
        cat.pData[i] = pData[i];      
      for (int i = 0; i < s.length; i++) 
       cat.pData[length + i] = s.pData[i];
      cat.pData[cat.length] = '\0';
    }

    char GCString::operator[] (int i) {
      return pData[i];
    }

    GCString GCString::operator+ (const char* cString) {
      GCString temp (cString);
      GCString cat;
      cat = operator+ (temp);
      return cat;
    }

    GCString GCString::operator+= (GCString& s) {
      gc_ptr<char> temp_string = gc_new <char> (length + s.length + 1);
      for (int i = 0; i < length; ++i)
        temp_string[i] = pData[i];
      for (int i = 0; i < s.length; ++i) 
        temp_string[i + length] = s[i];
      temp_string[length + s.length] = '\0';
      pData = temp_string;
      length = length + s.length;
      return *this;
    }
\end{lstlisting}
