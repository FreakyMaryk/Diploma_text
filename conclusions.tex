\section{Заключение}

Следующие результаты были достигнуты в рамках данной дипломной работы. 

\begin{enumerate}
\item Были изучены и  описаны следующие подходы к неконсервативной и консервативной сборке мусора: [лучше через зпт, а не списком] 1, 2, 3. Также была составлена сравнительная характеристика предоставленных пользовательских уровней для различных языков и сборщиков мусора;

\item На основе проведенного анализа подходов к сборке мусора и сравнительной характеристики, было предложено разработать неконсервативный сборщик мусора для С++ с использованием алгоритма mark\&sweep;
\item Предложенное решение было реализовано на языках С/С++ в рамках проекта JBGC, и были продемонстрированы возможности пользовательского уровня получившегося неконсервативного сборщика мусора;

\item Был проведен сравнительный анализ языковых примитивов языка С++ и набора различных конструкций, предлагаемых к использованию в JBGC;
\end{enumerate}

\vspace{0.3cm}
Данный анализ показал, что текущих наработок достаточно для реализации большинства возможных языковых ситуаций. Но, в конечном счёте, в JBGC всё же существует ряд недоработок и ограничений, с которыми нужно бороться, таких как:
\begin{enumerate}
\item 
\end{enumerate}