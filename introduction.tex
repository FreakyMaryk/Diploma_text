\setcounter{page}{4}
\section*{Введение}
\addcontentsline{toc}{section}{Введение}

Динамическая память является одним из самых важных ресурсов, которым может распоряжаться программа. Использование
динамической памяти позволяет добиться выской гибкости и масштабируемости реализации, а также повысить уровень
абстракции и переиспользования кода. С другой стороны, использование динамической памяти ``вручную'' с помощью
примитивов низкого уровня может приводить к трудно обнаруживаемым, отложенным и редко проявляющимся ошибкам. 
Безопасное управление динамической памятью является одной из самых важных задач прикладного и системного программирования. 
Поэтому автоматизация управления динамической памятью~--- актуальная задача, решение которой так или иначе необходимо при
разработке многих современных языков программирования и окружающего их инструментария.

Одним из наиболее универсальных способов автоматического управления динамической памятью является \emph{сборка мусора} 
(garbage collection)~\cite{GCBook}. Суть сборки мусора заключается в том, что выделенная в приложении память, 
которая ему в дальнейшем не понадобится, освобождается системой управления памятью автоматически. Для этого в 
определённые моменты запускается процесс, самостоятельно освобождающий те участки памяти, до которых невозможно
добраться, используя ссылки, с которыми в данных момент работает программа.
 
Впервые сборка мусора была применена еще  в 1959 году Джоном Маккарти. Он использовал ее в среде программирования для 
разработанного им языка функционального программирования Lisp\footnote{\url{http://www.gnu.org/software/emacs/manual/html_node/elisp/Garbage-Collection.html}}.
Вначале сборку мусора стали применять преимущественно в функциональных и логических языках, поскольку в таких языках затруднено
отслеживание времени жизни объектов в памяти. В промышленных процедурных и объектно-ориентированных языках технология сборки мусора стала 
приобретать популярность лишь со второй половины 1980-х годов. До этого времени ручное управление памятью считалось предпочтительнее как более 
эффективное и предсказуемое. Со второй половины 1990-х годов все чаще и чаще механизм сборки мусора включают в языки и среды, ориентированные 
на прикладное программирование. После появления языка программирования Java\footnote{\url{https://www.java.com/ru}} в 1995 году 
сборка мусора стала настоящим ``мейнстримом''.  На данный момент технология сборки мусора используется в таких языках, как Java, C\#, 
Python, Ruby, Perl и многих других. 

Сборка мусора обладает как достоинствами, так и недостатками. По сравнению с ручным управлением, автоматическое 
управление памятью безопаснее: программисту не нужно заботиться о том, когда освобождать память из-под объектов. 
Это дает гарантию того, что не возникнет некоторых ошибок, таких как висячий указатель, т.е. указатель на уже 
освобожденный объект, или ошибка повторного освобождения памяти, когда программа пытается освободить память, 
которая уже была освобождена.

С другой стороны при использовании сборки мусора могут возникнуть следующие проблемы:

\begin{enumerate}
\item если на объект есть ссылки из других достижимых объектов, то он никогда не будет удален;
\item во время работы программы из-за запуска сборщика мусора возникают паузы в непредвиденные моменты времени, а их 
продолжительность не определена.
\end{enumerate}

Для того, чтобы сборка мусора была возможна, должны выполняться определенные условия. Во-первых, должна присутствовать 
возможность определить все указатели из любого объекта на другие элементы кучи. Во-вторых, не должно быть никаких 
операций над указателями (логических, арифметических и т.п.) Часть из этих условий может быть автоматически выполнена
для некоторых языков (например, языков, в которых нет явного понятия указателя), выполнение остальных должно быть
обеспечено компилятором и библиотекой поддержки времени исполнения. Если это по каким-то причинам невозможно, то, строго
говоря, невозможна и надежная сборка мусора в общем случае. Тем не менее даже тогда можно добиться правильной
работы сборщика мусора, если каким-то образом ограничить вид программ, для которых он запускается. Такой подход применяется 
в ситуации, когда требуется использовать сборку мусора для тех языков, для которых она невозможна (или труднореализуема) 
в общем случае. Таким языком, например, является С++.

Язык С++~--- один из самых развитых и распространенных универсальных языков программирования. Свою популярность он приобрел 
благодаря гибкости, пригодности для реализации широкого класса задач и эффективности кода, порождаемого его 
компиляторами. Однако оборотной стороной этих его достоинств является то, что он по принципиальным соображениям не 
совместим со сборкой мусора, поскольку содержит низкоуровневые конструкции оперирования с памятью (указатели, адресную 
арифметику, неконтролируемые приведения типов). Тем не менее широкий класс прикладных программ может обойтись без этих
низкоуровневых возможностей, поэтому для него существуют библиотечные инструменты, позволяющие автоматизировать
управление динамической памятью при соблюдении определенных соглашений. В данной дипломной работе описывается
библиотека, независимая от конкретного компилятора, применение которой позволяет использовать в прикладной
программе на С++ точную сборку мусора.
