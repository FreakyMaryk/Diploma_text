\setcounter{page}{4}
\section*{Введение}
\addcontentsline{toc}{section}{Введение}
Программисту важно писать  код свободный от ошибок. Наиболее уязвимым местом в коде с точки зрения человеческого фактора является ручное управление динамической памятью.  Безопасное управление памятью значительно повышает надёжность программной системы, исключая многие распространённые ошибки, которые могут возникать из-за обычной опечатки и требовать длительного поиска, а зачастую  -  оставаться вообще неисправленными. Поэтому в любом современном языке программирования и окружающем его инструментарии используется технология сборки мусора.

Этот механизм обладает рядом важных преимуществ над ручным управлением памяти, одно из которых -- безопасноть. Под безопасным управлением памятью подразумеваются средства языка программирования и его системы времени выполнения, которые гарантируют защиту от ошибок при работе с динамической памятью.
 
Сборка мусора\footnote{URL: \url{http://www.stroustrup.com/C++11FAQ.html}} — это один из споcобов автоматического управления динамической памятью. Суть сборки мусора заключается в том, что выделенная в приложении память, которая ему в дальнейшем не понадобится, освобождается системой управления памятью автоматически. Для этого в определённые моменты запускается процесс, называемый сборкой мусора.
 
Сборка мусора обладает как достоинствами, так и недостатками. По сравнению с ручным управлением, автоматическое управление памятью безопаснее: программисту не нужно заботиться о том, когда освобождать память из-под объектов. Это дает гарантию того, что не возникнет некоторых ошибок, таких как: висячий указатель, т.е. указатель на уже освобожденный объект, или ошибка повторного освобождения памяти, когда программа пытается освободить память, которая уже была освобождена.

С другой стороны при автоматическом управлении памятью могут возникнуть следующие проблемы:
\begin{enumerate}
\item Если на объект есть ссылки из других достижимых объектов, то он никогда не будет удален.
\item Во время работы программы из-за прогресса сборщика мусора возникают паузы в случайные моменты времени, а их продолжительность не определена.
\item  Должны быть выполнены требования для реализации сборщика мусора.
\end{enumerate}
Во-первых, должна присутствовать возможность определить все указатели из любого объекта на другие элементы кучи.
Во-вторых, не должно быть никаких операций над указателями (логических, арифметических и т.п.)
Помимо этого, приходится затрачивать время на отслеживание дополнительной информации по объектам.


\pagebreak

Помимо этого, приходится затрачивать время на отслеживание дополнительной информации по объектам.
 
Впервые сборка мусора была применена еще  в 1959 году Джоном Маккартни. Он использовал ее в среде программирования для разработанного им языка функционального программирования Lisp.\footnote{URL:   \url{http://www.gnu.org/software/emacs/manual/html_node/elisp/Garbage-Collection.html}} В дальнейшем сборку мусора стали применять преимущественно в функциональных и логических языках. Необходимость сборки мусора в языках этих типов обусловлена тем, что структура таких языков делает крайне неудобным отслеживание времени жизни объектов в памяти и ручное управление ею .
 
В промышленных процедурных и объектно-ориентированных языках технология сборки мусора стала приобретать популярность лишь со второй половины 1980-х годов. До этого времени ручное управление памятью считалось предпочтительнее, как более эффективное и безопасное. Со второй половины 1990-х годов все чаще и чаще механизм сборки мусора включали в языки и среды, ориентированные на прикладное программирование. После появляния языка программирования Java\footnote{URL: \url{https://www.java.com/ru/}} в 1995 году, сборка мусора стала настоящим “мейнстримом”.  На данный момент, технология сборки мусора используется в таких языках, как Java, C\#, Python, Ruby, Perl и т.д.
Язык С++ является одним из самых сложных языков программирования. Свою популярность он приобрел благодаря своей гибкости, пригодности для программирования шикрокого класса задач. Несмотря на то , что язык С++ появился более 30 лет назад, до сих пор в нем не существует стандартных  технологий сборки мусора. По этой причине программисты используют  малоэффективные средства автоматического управления динамической памятью, такие как смартпойнтеры, счетчики ссылок, пулы и прочие .

Данная дипломная работа посвящена вопросу о том, как организовать сборку мусора для С++ так, чтобы она была неконсервативной (точной) и  поддерживала языковые примитивы C++.